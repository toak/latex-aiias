\section{Summary}
% Finding the golden thread:  Summarize the argument in about two pages.  Focus on the main thesis of the section, how it relates to the topic of the whole book, and how the author proceeds to make his/her case as the section progresses. Lengthy summaries are unnecessary. Look for the links between ideas and how those ideas move the argument along, leading to the conclusion.  In the case of an edited book, discover the rationale for grouping these articles together and show how, taken together, they provide a perspective or contrasting perspectives on a particular topic.  While you may compare points of view in your account, do not summarize each article individually. 
\cite{roberts2015developing}

\section{Internal critique}
% Show how the reading holds together.  Is it well organized?  Is the argument sound? Is it well researched and documented? Are the conclusions warranted?  Does the writing style fit the content? Does the section fit well with the whole book? Internal analysis of the text should be done in one to two pages.
\cite{blanchard1998transforming}

\section{External critique}
% What are others in the field saying about this topic?  Bring into dialogue the argument of the reading with other voices. A book review of the web might be helpful.  This section should take about one page.
\cite{bachmann2017ethical}

\section{My critique and application}
% Your critique/application: Based on your knowledge and experience, what is your opinion of this work (these works)?  How the understanding of this text has given you aid in your own thinking and action? Use about one page for this analysis.  
\cite{blackaby2011spiritual}

\section{Questions for discussion}
% Compose five questions based on the content of the book that will stimulate discussion on the issues raised in the readings.

\begin{enumerate}
\item first
\item second
\item third
\item forth
\item fifth
\end{enumerate}
